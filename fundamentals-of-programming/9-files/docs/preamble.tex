\usepackage{fontspec}

\usepackage{polyglossia}
\setdefaultlanguage{ukrainian}

% Встановлюємо Times New Roman як основний шрифт
\setmainfont{Times New Roman}
\newfontfamily\cyrillicfont{Times New Roman}
\newfontfamily\cyrillicfonttt{JetBrainsMono Nerd Font}

\usepackage{geometry}
\geometry{top=25mm,bottom=15mm,left=25mm,right=10mm} % Встановлення полів сторінки

\usepackage{titlesec}
\titleformat{\section}{\normalfont\large\bfseries\centering}{}{0pt}{}
\titleformat{\subsection}{\normalfont\bfseries\centering}{}{0pt}{}
\titleformat{\subsubsection}{\normalfont\bfseries\itshape}{}{0pt}{}

\setlength{\parindent}{12.5mm}  % Встановлення загального абзацного відступу
\newcommand{\address}{\hspace*{90mm}} % Відступ для адресата
\newcommand{\approval}{\hspace*{100mm}} % Відступ для «Гриф затвердження»
\newcommand{\signature}{\hspace*{125mm}} % Відступ для підпису

\usepackage{setspace}

\pagenumbering{arabic} % Встановлення арабської нумерації

\usepackage{amsmath} % Для розширених математичних виразів
\usepackage{amsthm} % Для середовища доказів

\usepackage{enumitem}
\setlist[enumerate,2]{label=\theenumi.\arabic*} % Нумерація = попередній.наступний рівень
\setlist[enumerate,3]{label=\theenumii.\arabic*} % Нумерація = попередній.попередній.наступний рівень

\usepackage{svg} % Для вставки блок-схем у форматі SVG
\usepackage{float} % Для керування розміщенням блок-схем


% Визначення кольорів, натхнених темою Solarized Light
\definecolor{background}{HTML}{FFFFFF} % Світлий кремовий
\definecolor{keyword}{HTML}{268BD2}    % М'який синій
\definecolor{string}{HTML}{2AA198}     % Бірюзовий
\definecolor{comment}{HTML}{93A1A1}    % Сірий
\definecolor{function}{HTML}{859900}   % Оливковий зелений
\definecolor{variable}{HTML}{B58900}   % Гірчичний жовтий
\definecolor{type}{HTML}{D33682}       % Пурпуровий

\usepackage{listings}       % Для підсвітки синтаксису
\usepackage{xcolor}       % Для роботи з кольорами

% Налаштування listings із кастомним шрифтом та кольоровою схемою
\lstset{
	basicstyle=\footnotesize\ttfamily,          % Використання \ttfamily для моноширинного шрифту
	backgroundcolor=\color{background},
	keywordstyle=\color{keyword}\bfseries,
	stringstyle=\color{string},
	commentstyle=\color{comment}\itshape,
	identifierstyle=\color{variable},
	emphstyle={\color{function}},
	numberstyle=\footnotesize\color{comment},
	numbers=left,
	numbersep=5pt,
	tabsize=4,
	frame=lines,
	breaklines=true,
	xleftmargin=10pt,    % Опційно: корекція лівого відступу
	xrightmargin=10pt,   % Опційно: корекція правого відступу
	% Налаштування ширини символів (зменшити при потребі)
	columns=fullflexible  % Дозволити більшу гнучкість у ширині символів
}

\usepackage{multicol}
\usepackage{pgffor}
