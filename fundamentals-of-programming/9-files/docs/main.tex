% !TEX program = lualatex
\documentclass[a4paper,12pt]{extarticle}
\usepackage{hyperref}

\usepackage{fontspec}

\usepackage{polyglossia}
\setdefaultlanguage{ukrainian}

% Встановлюємо Times New Roman як основний шрифт
\setmainfont{Times New Roman}
\newfontfamily\cyrillicfont{Times New Roman}
\newfontfamily\cyrillicfonttt{JetBrainsMono Nerd Font}

\usepackage{geometry}
\geometry{top=25mm,bottom=15mm,left=25mm,right=10mm} % Встановлення полів сторінки

\usepackage{titlesec}
\titleformat{\section}{\normalfont\large\bfseries\centering}{}{0pt}{}
\titleformat{\subsection}{\normalfont\bfseries\centering}{}{0pt}{}
\titleformat{\subsubsection}{\normalfont\bfseries\itshape}{}{0pt}{}

\setlength{\parindent}{12.5mm}  % Встановлення загального абзацного відступу
\newcommand{\address}{\hspace*{90mm}} % Відступ для адресата
\newcommand{\approval}{\hspace*{100mm}} % Відступ для «Гриф затвердження»
\newcommand{\signature}{\hspace*{125mm}} % Відступ для підпису

\usepackage{setspace}

\pagenumbering{arabic} % Встановлення арабської нумерації

\usepackage{amsmath} % Для розширених математичних виразів
\usepackage{amsthm} % Для середовища доказів

\usepackage{enumitem}
\setlist[enumerate,2]{label=\theenumi.\arabic*} % Нумерація = попередній.наступний рівень
\setlist[enumerate,3]{label=\theenumii.\arabic*} % Нумерація = попередній.попередній.наступний рівень

\usepackage{svg} % Для вставки блок-схем у форматі SVG
\usepackage{float} % Для керування розміщенням блок-схем


% Визначення кольорів, натхнених темою Solarized Light
\definecolor{background}{HTML}{FFFFFF} % Світлий кремовий
\definecolor{keyword}{HTML}{268BD2}    % М'який синій
\definecolor{string}{HTML}{2AA198}     % Бірюзовий
\definecolor{comment}{HTML}{93A1A1}    % Сірий
\definecolor{function}{HTML}{859900}   % Оливковий зелений
\definecolor{variable}{HTML}{B58900}   % Гірчичний жовтий
\definecolor{type}{HTML}{D33682}       % Пурпуровий

\usepackage{listings}       % Для підсвітки синтаксису
\usepackage{xcolor}       % Для роботи з кольорами

% Налаштування listings із кастомним шрифтом та кольоровою схемою
\lstset{
	basicstyle=\footnotesize\ttfamily,          % Використання \ttfamily для моноширинного шрифту
	backgroundcolor=\color{background},
	keywordstyle=\color{keyword}\bfseries,
	stringstyle=\color{string},
	commentstyle=\color{comment}\itshape,
	identifierstyle=\color{variable},
	emphstyle={\color{function}},
	numberstyle=\footnotesize\color{comment},
	numbers=left,
	numbersep=5pt,
	tabsize=4,
	frame=lines,
	breaklines=true,
	xleftmargin=10pt,    % Опційно: корекція лівого відступу
	xrightmargin=10pt,   % Опційно: корекція правого відступу
	% Налаштування ширини символів (зменшити при потребі)
	columns=fullflexible  % Дозволити більшу гнучкість у ширині символів
}

\usepackage{multicol}
\usepackage{pgffor}


\newcommand{\labnumber}{9}
\newcommand{\labtopic}{Робота з файлами.}
\newcommand{\labtask}{Написати програму, яка виконує наступні дії:}
\newcommand{\labdate}{«2» січня 2025 р.}

\newcommand{\includelistings}[3]{%
	\foreach \filename in {#1} {%
		\subsubsection{#2/\filename}%
		\lstinputlisting[language=#3]{#2/\filename}%
	}%
}

\newcommand{\img}[1]{%
  \resizebox{\textwidth}{!}{\includegraphics{img/#1}}\\%
}

\begin{document}

\doublespacing
\begin{titlepage}
	\begin{center}
		{\fontsize{14pt}{16pt}\selectfont\textbf{КПІ ім. Ігоря Сікорського} \\
			\textbf{Факультет інформатики та обчислювальної техніки} \\
			\textbf{Кафедра інформатики та програмної інженерії}}\\
	\end{center}

	\vspace{1cm}

	\begin{center}
		{\fontsize{14}{16pt}\selectfont\textbf{Звіт до комп'ютерного практикуму з курсу}\\
			\textbf{«Основи програмування»}\\}
	\end{center}

	\vspace{6cm}

	\singlespacing
	\begin{multicols}{3}
		{
			\raggedright
			\fontsize{12pt}{12pt} \selectfont
			Прийняв \\
			асистент кафедри ІПІ \\
			Ахаладзе А. Е. \\
			\labdate
		}

		\columnbreak
		\vfill\null
		\columnbreak

		{
			\raggedright
			\fontsize{12pt}{14pt} \selectfont
			Виконав \\
			студент групи ІП-43 \\
			Дутов І. А. \\
		}
	\end{multicols}

	\vfill
	\begin{center}
		{\fontsize{14}{16}\selectfont\textbf{Київ 2024}}
	\end{center}
\end{titlepage}
\newpage

\singlespacing

\section{Комп'ютерний практикум №\labnumber}
\vspace{-10pt}
\begin{center}
	\emph{\textbf{Тема:} \labtopic}
\end{center}
\vspace{-5pt}

\textbf{Завдання:}\\
\par\labtask
\begin{enumerate}
	\item Створення файлу.
	\item Відкриття вже створеного файлу та завантаження даних із файлу.
	\item Видалення файлу.
	\item Запис в файл даних, введених з клавіатури користувачем.
	\item Зчитування запису(ів) із файлу і виведення їх на екран.
	\item Редагування запису з файлу.
	\item Впорядкування (на вибір користувача, за зростанням або спаданням) записів в файлі за полями: назва області, площа, кількість населення.
	\item Вставка у впорядкований файл записів так, щоб файл залишився впорядкованим.
	\item Видалення запису з файлу.
\end{enumerate}

\subsection{Текст програми}
\includelistings{CMakeLists.txt}{..}{bash}
\includelistings{main.c}{../src}{c}
\includelistings{common.h, fileContext.c, sort.c, stack.c}{../src/common}{c}
\includelistings{menu.h, default.c, menu.c}{../src/menu}{c}
\includelistings{actions.h, files.c, records.c, misc.c, regions.c, utils.h, utils.c}{../src/actions}{c}
\includelistings{io.h, number.c, string.c, choices.c, validators.c, utils.h, utils.c}{../src/io}{c}

\subsection{Введені та одержані результати}

\subsubsection{Створення файлу}
\img{create_file.png}
\subsubsection{Створення записів}
\img{add_record.png}
\subsubsection{Зчитування всіх записів}
\img{read_all_records.png}
\subsubsection{Видалення запису}
\img{delete_record.png}
\subsubsection{Редагування запису}
\img{edit_record.png}
\subsubsection{Зчитування запису}
\img{read_record.png}
\subsubsection{Впорядкування записів}
\img{sort_name_ascending.png}\\
\img{sort_area_descending.png}\\
\img{sort_population_ascending.png}
\subsubsection{Вставка у впорядкований файл} % TODO: Do I need to show all the error conditions? 
\img{insert_record}
\subsubsection{Видалення файлу} % TODO: Do I need to show my filesystem?
\img{delete_file.png}
\subsubsection{Файл без сигнатури}
\img{error_signature.png}
\subsubsection{Неправильне ім'я файла}
\img{errors_filename.png}
\subsubsection{Неправильно відформатований вміст файла}
\img{errors_reading.png}\\
\img{errors_file_contents.png}
\subsubsection{Використання невідсортованого файлу для вставки}
\img{file_without_sort1.png}\\
\img{file_without_sort2.png}
\subsubsection{Читання, редагування, видалення записів, що не існують}
\img{errors_without_file.png}


\emph{\textbf{Висновки:}} Програма працює коректно. Програма вирішує поставлене завдання.

\end{document}
